% arara: xelatex
% arara: xelatex
% arara: xelatex

% options:
% thesis=B bachelor's thesis
% thesis=M master's thesis
% czech thesis in Czech language
% english thesis in English language
% hidelinks remove colour boxes around hyperlinks

\documentclass[thesis=M,english]{FITthesis}[2019/12/23]

\usepackage[utf8]{inputenc}

\usepackage{graphicx} %graphics files inclusion
\usepackage{wrapfig}
\usepackage{amsmath} %advanced maths
\usepackage{amssymb}
\usepackage{dirtree} %directory tree visualisation
\usepackage{cleveref} % intelligent crossreference
\usepackage{hyperref} % cross-reference
\usepackage{afterpage}
\usepackage{notoccite} % prevent TOC cite 
\usepackage{xcolor}
\usepackage{blindtext}
\usepackage{enumitem}
\usepackage{tabularx} % tables
\usepackage{minted} % code highlight
\usemintedstyle{friendly}

% % list of acronyms
\usepackage[acronym,nonumberlist,toc,numberedsection=autolabel]{glossaries}
\iflanguage{czech}{\renewcommand*{\acronymname}{Seznam pou{\v z}it{\' y}ch zkratek}}{}
\makeglossaries

% Used from http://www.herout.net/blog/2017/03/pomalu-uz-pojdme-psat/
\newglossaryentry{formula}
{
        name=formula,
        description={A mathematical expression}
}
\newcommand{\todo}[1]{\textcolor{red}{\textbf{[#1]}}}
\newcommand{\blind}[1][1]{\textcolor{gray}{\Blindtext[#1][1]}}
\newcommand{\RNum}[1]{\uppercase\expandafter{\romannumeral #1\relax}}

\setlength{\fboxsep}{0.005pt}
\newcommand{\tmpframe}[1]{\fbox{#1}}
%\renewcommand{\tmpframe}[1]{#1} ENABLE WHEN PRINTING


\newcommand{\newpara}
{
  \vskip 0.2in
}

% Q.A env

\newenvironment{questions}{\begin{enumerate}[label=\bfseries\arabic*.]\bfseries}
                      {\end{enumerate}}
\newenvironment{answer}{\par\normalfont}{}

% % % % % % % % % % % % % % % % % % % % % % % % % % % % % % 


\department{Department of Software Engineering}
\title{Coffee Time Mobile Application in Flutter}
\authorGN{Petr} %(křestní) jméno (jména) autora
\authorFN{Nymsa} %příjmení autora
\authorWithDegrees{Bc. Petr Nymsa} %jméno autora včetně akademických titulů
\author{Petr Nymsa} %jméno autora bez akademických titulů
\supervisor{Ing. David Šenkýř}
\acknowledgements{ \todo{todo}Děkuji všem a za vše. Nevíte-li, co sem napsat, příkaz odstraňte.}

\abstractCS{Práce se zabývá využitím multiplatformního frameworku Flutter, umožňující tvorbu aplikací pro mobilní systémy, web i desktop systémy. Hlavním tématem je jeho fungování a~použitelnost při tvorbě aplikací. Flutter je prakticky vyzkoušen na návrhu~a~implementaci aplikace Coffee Time pro operační systém Android a~iOS. Aplikace uživatelům pomáhá vyhledávat kavárny v blízkém okolí s~ možností filtrování dle různých kritérií. 

Práce je zaměřena především na využití frameworku, následný návrh a~implementaci ukázkové aplikace. Kromě použitého frameworku Flutter, aplikace využívá i~serverless přístup pro vývoj serverové části. TODO }

\abstractEN{todo}

\placeForDeclarationOfAuthenticity{Prague}
\declarationOfAuthenticityOption{4} %volba Prohlášení //TODO

\keywordsCS{Flutter framework, reaktivní programování, Android aplikace, vyhledávač kaváren, serverless, Firebase.}

\keywordsEN{Flutter framework, reactive programming, Android application, cafe search, serverless, Firebase.}

\website{http://TODO} %volitelná URL práce, objeví se v tiráži


\begin{document}

% Acronyms and glossary
\newglossaryentry{formula}
{
        name=formula,
        description={A mathematical expression}
}
\newacronym{aot}{AOT}{Ahead-of-time}
\newacronym{bloc}{BLoC}{Business Logic Component}
\newacronym{ctu}{CTU}{Czech Technical University}
\newacronym{cta}{CTA}{Coffee Time API}
\newacronym{gpa}{GPA}{Google Places API}
\newacronym{jit}{JIT}{Just-in-time}
\newacronym{lofi}{Lo-Fi}{Low Fidelity}
\newacronym{hifi}{Hi-Fi}{High Fidelity}
\newacronym{ui}{UI}{User Interface}

\glsresetall
% % % % % % % % % % % % % % % % % % % % % % % % % % 

\begin{introduction}
	Doplňte úvod Vaší práce.  \gls{ctu}
	
	\todo{TODO}
	\blind[2]
	
    \begin{figure}
        \centering
        \tmpframe{\includegraphics[width=0.5\linewidth]{img/TODO-image.pdf}}
        \caption{One image. \todo{Napsat pořádný titulek}}
        \label{fig:TODO}
    \end{figure}
	
\end{introduction}

\chapter{Flutter Foundations}
\label{ch:flutter}
In the introduction, a new, promising, cross-platform framework was introduced. The Flutter's primary goal is to provide the ability to build high-performance, high-fidelity apps for \textit{iOS}, \textit{Android}, web, and desktop from a single code-base~\cite{flutter-technical-overview}. In this chapter, the framework's philosophy will be described. Used programming language and theory of reactive programming is briefly introduced. The chapter describes the concept of widgets as a base building block for every application. Later on, one of the most critical topic -- state management is discussed, its existing approaches and which to prefer when building applications. At the end of this chapter, the brief look under the framework's hood is discussed.

Flutter includes a modern react-style framework, a 2D rendering engine, predefined widgets and development tools. The primary premise is a motto "everything is a widget". A widget is an immutable building block of application which is part of the user interface. Each widget can define structural elements such as button, stylistic elements such as colour or it can define the interface's layout, such as padding. Widgets are composed as a tree hierarchy with composing each widget to another. If any event occurs (such as user interaction), the framework can rebuild part of this tree to redraw the screen.  

\begin{figure}[htp]
    \centering
    \includegraphics[width=0.5\linewidth]{img/flutter/hello-flutter.pdf}
    \caption{Widget composition example}
    \label{fig:hello-flutter}
\end{figure}

Flutter encourages developers to create and use small, single-purpose widgets and compose them to create complex interfaces and layouts. Take an~example from~\cref{listing:hello-flutter} , where, the root widget, Container is used to create a rectangular visual element. The Container is something like <div> element in HTML.  Under the Container is Column widget which composes children widgets into the vertical direction. Finally, Text widget displaying text "Hello Flutter" and Icon widget showing star icon. The~composition hierarchy along with a~result is shown at~\cref{fig:hello-flutter}.

\begin{listing}[ht]
\begin{minted}{dart}
Container(
    padding: const EdgeInsets.all(5.0),
    child: Column(
      mainAxisAlignment: MainAxisAlignment.center,
      children: [
        Text('Hello Flutter'),
        Icon(Icons.star),
      ],
    ),
),
\end{minted}
\caption{Widget composition code example}
\label{listing:hello-flutter}
\end{listing}
% ----- % ----- % ----- % ----- % ----- % ----- % ----- % ----- % ----- % ----- % ----- % ----- % ----- % ----- %
\section{Technical overview}
Flutter uses programming language Dart (specification v2.0~\cite{dart-specs}) also made by Google and is inspired by languages such as JavaScript. Dart using statically typed system with runtime checks, but like many other languages highly use type inference~\cite{dart-type-system}. Dart can be used from writing simple scripts to full-featured applications. The Dart has flexible compiler technology which can decide running code in different ways, depending on the targeted platform~\cite{dart-platforms}. 

\begin{itemize}
    \item \textbf{Dart Native} -- For programs targeting devices (mobile, desktop, server, and more), Dart Native includes both a Dart VM with \gls{jit} compilation and an~\gls{aot} compiler for producing machine code.
    \item \textbf{Dart Web} -- For programs targeting the web, Dart Web includes both a development time compiler (dartdevc) and a production time compiler (dart2js).
\end{itemize}

\begin{figure}[htp]
    \centering
    \includegraphics[width=0.8\linewidth]{img/flutter/dart-platforms.pdf}
    \caption{Dart platforms~\cite{dart-platforms}}
    \label{fig:dart-platform}
\end{figure}

Flutter performs the use of both ways. If the targeted platform is web, the \textit{Dart~Web} is used. For other platforms \textit{Dart~Native} is chosen. The Dart~Native's \gls{jit} compilation is highly used to support fast development process with ``hot-reload'' functionality. Then the~\gls{aot} compilation is used for the best-optimised production-ready result on the native platform.  

The Flutter framework is organised into several layers (see~\cref{fig:flutter-layer-cake}), where each layer makes usage of the previous one. The upper layers are more frequently used by developers on a daily basis, and lower layers are used only if the developers need to create particular customisations. 

\begin{figure}[htp]
    \centering
    \includegraphics[width=0.8\linewidth]{img/flutter/flutter-layer-cake.png}
    \caption{Flutter system overview~\cite{flutter-technical-overview}}
    \label{fig:flutter-layer-cake}
\end{figure}

Unlike the other frameworks, Flutter uses high-performance 2D rendering engine and draws everything onto the screen directly. That means pixel-perfect control over what and how it is displayed. The most top layer, Material and Cupertino, are set of widgets which defines Material Design (Android systems) and Apple Design component respectively. To highlight that, Flutter does not use native components, but everything draws by himself. These two layers support developers to bring the standardised look and feel to the targeted platform. \todo{review}

% --- # --- # --- # --- # --- # --- # --- # --- # --- # --- # --- # --- # --- # --- # --- # --- # --- # --- #
\subsection{Reactive Programming}
The Flutter makes significant usage from the concept of reactive programming. There is nearly always a~requirement to update data in response to user interaction or any other event such as getting data from the~ server. More than that, sometimes it is necessary to update different parts of the user interface in response to these events. 

The Flutter creates user interface by composing \textbf{immutable} widgets. The immutability is the key point here. Whenever user interface needs to ``redraw'' screen, the~part of the~widget tree is replaced by \textbf{new} widget instances (in fact, it is not simple as that, and this topic is more deeply discussed later in this chapter \todo{remove if not}). In many other \gls{ui} frameworks, such as \textit{Xamarin}, is usually taken the approach of coupling \gls{ui} components with view-models through concepts such as data binding~\cite{xamarin-data-binding}. That means that whenever \gls{ui} needs to~change, the~components mutate application's state. Flutter takes an~entirely different approach. It can be said ``here is the current state of the application, draw something on the screen accordingly''.

\subsubsection{The Notion of Streams}
A Stream can be described as ``a pipe with two ends, only one allowing to insert something into it. When something is inserted into the pipe, it flows inside the pipe and goes out by the other end''~\cite{reactive-didier}. The~Stream can convey any~data type, from simple values to~events, complex object or even another stream. The~data can come to the~Stream, for example, from an external data source such as server connection or from events such as user interactions. In Dart, the Streams support manipulating them, filtering, re-grouping, modify data before they are send and much more. This functionality can be used to build reactive \gls{ui}. The~Flutter has several widgets supporting streams to rebuild part of the~\gls{ui} whenever new data arrived into the~Stream.

The answer to the question ``what is reactive programming`` could be ``Reactive programming is programming with asynchronous data streams``~\cite{reactive-didier},\cite{reactive-red-hat}. Within Flutter framework, anything from an interaction event (tap, gesture), changes on a variable, messages, everything that may change is conveyed and triggered by streams.

That means that with reactive programming, according to~\cite{reactive-didier}, the~application:

\begin{quote}
    \begin{itemize}
        \item becomes asynchronous
        \item is architectured around the notion of Streams and their listeners
        \item when something happens somewhere (an event, a change of a variable) a~notification is sent to a~Stream
        \item if ``somebody'' listens to that Stream, it will be notified and will take appropriate action(s), whatever its location in the~application
    \end{itemize}
    
    From Widgets perspective -- Widget does not longer need to know
    
    \begin{itemize}
        \item what is going to happen next,
        \item who might use this information (no one, one or several Widgets)
        \item where this information might be used (nowhere, same screen, another one, several ones)
        \item when this information might be used (almost directly, after several seconds, never)
    \end{itemize}
\end{quote}

Later on in this chapter, the pattern Business Logic Component (BLoC) is introduced. This pattern uses Streams to manage application life-cycle and state management. 
% ----- % ----- % ----- % ----- % ----- % ----- % ----- % ----- % ----- % ----- % ----- % ----- % ----- % ----- %
\section{Everything Is a Widget}
In this section, we will discuss in more details how the UI is built. Every UI consists of the layout and individual components. The layout defines the screen's base structure, such as a menu on the top and subsequent actions in the bottom. Then the layout is composed of individual components, such as menu, buttons or icons. Together they create a final interface.

These building blocks in the Flutter are called ``Widgets''. Whatever it is simple text, button or complex parts of the~layout, such as grid with multiple columns and rows -- \textbf{everything is a widget}.  Widgets describe what their view should look like given their current configuration and state. When a widget's state changes, the widget rebuilds its description, which the framework diffs against the previous description in order to determine the minimal changes needed in the underlying render tree to transition from one state to the next~\cite{flutter-widget-intro}. As the~composition to the~tree implies, each widget has at most one parent and zero or more children widgets. This tree, called ``widget tree'', is in fact, one of the~three trees involved. The~framework has a sophisticated way of decision about how the~trees should be rebuilt and screen updated. This behaviour is in details described later in this chapter \todo{if not, remove}.

Flutter framework uses only one language to define both the~user interface and business logic as well.  Widgets are Dart class which inherits from some of the widget's base class (typically \textit{StatelessWidget} or \textit{StatefulWidget}). Each widget has a build method which defines how the widget should be built (and drawn on the screen). 
% --- # --- # --- # --- # --- # --- # --- # --- # --- # --- # --- # --- # --- # --- # --- # --- # --- # --- #
\subsection{Widget Are Not Only Visible Parts}
Widgets are not only visible parts of the UI such as clickable buttons, text or icons. The widgets also define layouts such as columns, rows, grids, the margin between other widgets, padding around them and more. 

\begin{figure}[htp]
    \centering
    \includegraphics[width=0.75\linewidth]{img/flutter/layout_compose.png}
    \caption{Compose widgets to create layout~\cite{flutter-widget-layout}}
    \label{fig:flutter-compose-widget}
\end{figure}

An example of widget composing creating a layout is shown at~\cref{fig:flutter-compose-widget}. The root widget, a \textit{Row} widget, contains two nested widgets. On the left is a \textit{Column} which contains more nested widgets and on the right, \textit{Image} widget which displays a product image.  The break-down of the left column widget can be seen at~\cref{fig:flutter-compose-widget-detail}. 

\begin{figure}[htp]
    \centering
    \includegraphics[width=0.75\linewidth]{img/flutter/layout_compose_detail.png}
    \caption{Compose widgets to create layout -- left column detailed~\cite{flutter-widget-layout}}
    \label{fig:flutter-compose-widget-detail}
\end{figure}
% --- # --- # --- # --- # --- # --- # --- # --- # --- # --- # --- # --- # --- # --- # --- # --- # --- # --- #
\subsection{Stateless vs Stateful Widget}
In the introduction was said that Flutter's approach of displaying current user interface is declarative -- ``here is the current state of the application, draw something on the screen accordingly''.  In the Flutter, whenever application' state changes, the user interface is redrawn. There is no imperative changing of the~\gls{ui}, such as \verb|textWidget.text = 'new text'|. The~advantage of the~declarative approach is that there is only one code path for any state of the~\gls{ui}. Developers just describe how the screen should look for a given state, and that is it~\cite{flutter-declarative}. The \gls{ui} can be described as a formula where \gls{ui} is equal to function which takes a state and returns new \gls{ui}~(\cref{fig:flutter-ui-formula}).

\begin{figure}[htp]
    \centering
    \includegraphics[width=0.6\linewidth]{img/flutter/ui_f_state.png}
    \caption{User interface formula~\cite{flutter-declarative}}
    \label{fig:flutter-ui-formula}
\end{figure}
% --- & --- & --- & --- & --- & --- & --- & --- & --- & --- & --- & --- & --- & --- & --- & --- & --- & --- &
\subsubsection{Build Context}
An~essential part of the widgets is \textit{BuildContext}. A~context is a~reference to the~location of the~widget within the~part of the~tree~\cite{notion-widget-didier}. Each widget has its own context. As~widgets are composed to the~tree, the contexts are as well. The widget has access to its own context and its parent context. 

The \textit{BuildContext} is provided to each widget through the~build method and is used to find the~widget's ancestors.  This is commonly used to obtain a~defined application theme or get a~reference to a~navigator widget, which is used to do navigation between screens. 
% --- & --- & --- & --- & --- & --- & --- & --- & --- & --- & --- & --- & --- & --- & --- & --- & --- & --- &
\subsubsection{Local vs Application State}
The state is anything that forms what should be displayed. The state is any data what are needed in order to rebuild \gls{ui} at the~moment~\cite{flutter-local-app-state}. The~state can be separated into two concepts -- local state and application state. 

\begin{itemize}
    \item \textbf{Local state} -- Local state is which can be tied into one widget. It can be, for example, current tab in the ``tab selector'' widget, current progress of animation or state of checkbox (checked or unchecked).
    \item \textbf{Application state} -- Application State is which can not be local, whenever some information is needed to share across multiple widgets, the~state which should be kept during a~user session. An~example of application state can be a~logged user information, loaded articles from the~server or chat messages.
\end{itemize}
% --- & --- & --- & --- & --- & --- & --- & --- & --- & --- & --- & --- & --- & --- & --- & --- & --- & --- &
\subsubsection{Stateless Widget}
A~Stateless Widget is a widget which does not manage its own state. Once it gets its parameters, and it was built through the build method, it cannot be changed. Remember, that whenever Flutter decides to redraw the~screen, part of the~tree is rebuilt, but with new instances of the~widgets. Typical examples of the Stateless Widgets can be Container, Text or Icon. These widgets accept many parameters which can alter their look (and behaviour), but they cannot be changed later on by~themselves. 
% --- & --- & --- & --- & --- & --- & --- & --- & --- & --- & --- & --- & --- & --- & --- & --- & --- & --- &
\subsubsection{Stateful Widget}
Whenever widget needs to manage its state and for example, for the response of an event wants to mutate its state, the widget should be stateful.  The widget as a~Stateless accepts parameters which can be used to configure this widget but also has an associated object, called state. This state object is an active part of the widget and is used to change widget and force framework rebuilt\gls{ui}.  An~example of a Stateful Widget can be a checkbox with ``checked'' state. 

Stateful Widget does not have only \verb|build| method but has associated State object which defines several methods to support widget's lifecycle. These methods are \verb|initState()| for any state initialisation and \verb|dispose()| to clear any allocated resources. 

The state object is associated with widget's \textit{BuildContext}. This association is permanent, and state object will never change it~\cite{notion-widget-didier}. Even if the~Widget Context can be moved around the~tree structure, the~state will remain associated with that context. This means as a~stateless widget, the~stateful widget itself can be replaced during tree rebuild, but the~state object is persisted. 
% --- & --- & --- & --- & --- & --- & --- & --- & --- & --- & --- & --- & --- & --- & --- & --- & --- & --- &
\subsubsection{Force Rebuild with setState()}
As was stated, Stateful widget can tell the framework to rebuild, and the widget can be redrawn based on changed state. The~stateful widget has method \verb|setState(callback)| which is used to do such rebuild. Inside \verb|callback| developer should change the~widget's state to the~new value and framework will rebuild the widget based on that~new state. 

\subsubsection{Case Study: Counter Application}
Suppose application where are two buttons. One button increments a value (\verb|counter|) and other decrements. The counter value is displayed within two Text widgets located on different places within the application. Whenever any of the buttons are clicked, and the value is changed, all Text widgets should reflect this change.

\begin{figure}[htp]
    \centering
    \includegraphics[width=0.33\linewidth]{img/flutter/counter_app_base.png}
    \caption{Counter application}
    \label{fig:counter-app}
\end{figure}

\begin{figure}[htp]
    \centering
    \includegraphics[width=0.40\linewidth]{img/flutter/counter-base.pdf}
    \caption{Counter application widget tree}
    \label{fig:counter-app-widget-tree}
\end{figure}

% \begin{figure}
%     \centering
%     \begin{minipage}{0.3\linewidth}
%         \centering
%         \includegraphics[width=0.40\linewidth]{img/flutter/counter_app_base.png}
%         \caption{Counter application}
%         \label{fig:counter-app}
%     \end{minipage}\hfill
%     \begin{minipage}{0.6\linewidth}
%         \centering
%         \includegraphics[width=0.40\linewidth]{img/flutter/counter-base.pdf}
%         \caption{Counter application widget tree}
%         \label{fig:counter-app-widget-tree}
%     \end{minipage}
% \end{figure}

Application's layout is shown at~\cref{fig:counter-app} and corresponding widget tree~\cref{fig:counter-app-widget-tree} (shortened for brevity). In the \textit{AppBar} and in the centre of the screen, is \textit{Text} widget which displays current value. On the bottom of the screen are \textit{FloatingActionButtons} widgets, which increments (decrements) counter value. The value needs to be accessible to the Text and to the button as well. Hence, the state is declared within the whole application's widget \textit{HomePage}.  

\begin{listing}[ht]
\begin{minted}{dart}
class HomePage extends StatefulWidget {
  HomePage({Key key}) : super(key: key);
  @override
  _HomePageState createState() => _HomePageState();
}
\end{minted}
\caption{HomePage widget definition}
\label{listing:counter-homepage-widget}
\end{listing}

The~\cref{listing:counter-homepage-widget} shows the definition of the \textit{HomePage} widget. The widget inherits from \verb|StatefulWidget| and declares \textit{HomePageState} which is an associated state object.

\begin{listing}[ht]
\begin{minted}{dart}
class _HomePageState extends State<HomePage> {
  int _counter = 0;

  void _incrementCounter() {
    setState(() {
      _counter++;
    });
  }

  void _decrementCounter() {
    setState(() {
      _counter--;
    });
  }

  @override
  Widget build(BuildContext context) {
    return Scaffold(
      appBar: AppBar(title: Text('Count $_counter')),
      body: Center(child: Text('Count: $_counter')),
      floatingActionButton: Column(
        mainAxisAlignment: MainAxisAlignment.end,
        children: <Widget>[
          FloatingActionButton(
            onPressed: _incrementCounter,
            child: Icon(Icons.add),
          ),
          FloatingActionButton(
            onPressed: _decrementCounter,
            child: Icon(Icons.remove),
          )]));
   }
}
\end{minted}
\caption{HomePageState -- setState example}
\label{listing:counter-base-state}
\end{listing}

\begin{figure}[htp]
    \centering
    \includegraphics[width=0.40\linewidth]{img/flutter/counter-base-setState.pdf}
    \caption{Expected rebuilt vs actual rebuilt}
    \label{fig:counter-app-base-build}
\end{figure}

The~\cref{listing:counter-base-state} shows definition of \textit{HomePageState}. Notice that \verb|build| method defines \textit{Scaffold} widget as the first widget. Scaffold widget creates the basic layout for Material applications with \textit{AppBar}, body and \textit{FloatingActionButtons}. \textit{HomePageState} defines \verb|int _counter = 0| variable, which is our state. There are also two private methods for incrementing and decrementing counter value. In each method, the \verb|setState()| is called with the appropriate state change.  These two methods are bound to \verb|onPressed| callback within \textit{FloatingActionButton}. 

The expectation is, that if the user pressed any button, the value is increment or decremented respectively. After that part of the~\gls{ui} is rebuild by the framework. This part should be two Text widgets which are interested in counter value. In fact, the state is defined within the~\textit{HomePage} widget, and so, the~whole \textit{HomePage} and its children are rebuilt~(\cref{fig:counter-app-base-build}). In~this small example, it is not really a~problem and performance should not be affected. However, if the~tree is deeply nested with heavy performance widgets (for example animations), it could lead to reduced performant application. 

How to define the application-wide state and prevent the necessary rebuilding part of the widget tree is a subject of ``state management'' section.
% ----- % ----- % ----- % ----- % ----- % ----- % ----- % ----- % ----- % ----- % ----- % ----- % ----- % ----- %
\section{State Management Approaches}
\todo{problem with setState() - define state at the top}
\todo{neccessart rebuilding}
\todo{callback hell}

\todo{inherited widget}

\todo{provider as a wrapper of inherited widget}
\todo{bloc solution}
% ----- % ----- % ----- % ----- % ----- % ----- % ----- % ----- % ----- % ----- % ----- % ----- % ----- % ----- %
\section{Native Features}
\todo{How flutter can use native functions, e.g camera. Flutter plugins.}
% ----- % ----- % ----- % ----- % ----- % ----- % ----- % ----- % ----- % ----- % ----- % ----- % ----- % ----- %
\section{Flutter Internals}
\todo{Flutter canvas engine. Skia framework. How flutter internally works. Widget and widget tree concept. Tree shaking.}
\todo{Usage of Keys to prevent rebuilds}
\todo{How Flutter rebuilds UI -- basic info about Widget Tree, Element Tree, State object, state comparison, const optimization}
\todo{animations? -- good example of rebuilding part of UI}




\chapter{Coffee Time Analysis}
% TODO Review ---------
In this chapter, the specification of the implemented application is outlined. The analysis of similar applications was made to obtain ideas.  After analysis, the low fidelity prototype was created to outline the first vision of the final application. Next, the high-fidelity prototype, along with Nielsen's heuristics analysis and user testing, were made. At the end of the chapter, considered tools and services are briefly described which are used to implement the application. 

\section{Considered application}

Coffee Time is an application focused on searching nearby cafes. Users should be able to search and find nearby cafes around them to decide which establishment to visit. Each cafe is displayed with information such as distance, user's reviews, photos, opening hours and more. 

Added value to this standard information is a feature so-called ``the tags''. These tags are user-added additional info which describes more precisely what given cafe offer or for example if that cafe allows pets inside. 

The set of tags is defined, and users can add these tags to the cafe.  Each tag can be reviewed by other users. These reviews are done through ``like'' and ``dislike'' functionality. The purpose of tag's reviews is to prevent outdated information or misleading information.  

The example of such a review can be ``User visited cafe which has tag 'dog friendly', but unfortunately this information was incorrect. Consequently, the user decided to open the application and review the cafe's tag 'dog friendly' with dislike.''

Together with likes and dislikes, each tag has a computed score.  Each like gives to score plus one and as an opposite, dislike minus one. 
If any tags reach the score to -4, the tag is removed from the cafe, more precisely is not shown anymore to users. 
If such removed tag is proposed by any user again, it obtains "like" review. Thus score is incremented to -3 and is shown again. 

The application is location-based and offers a map view to support the convenience user experience. Google Places API as a data source is considered. Any cafe can be marked as user's favourite to give a faster way to find cafe whenever the user wants. 

To conclusion, Coffee Time is application focused on one domain -- searching nearby cafes where to go to study, talk with friends or for example have a great coffee. The application should be simple to use with a clean user interface. 

\subsection{Use cases}
\todo{use cases, use case scenarios}
From the specification above, the use cases and use case scenarios were formed. The use cases diagram is listed in \cref{fig:use_case} and shows every use cases from the user perspective. Technically there is the role of ``application administrator'' which can check control back-end or available application's tags, but it skipped due to not importance of application perspective. 

\begin{figure}[htp]
    \centering
    \includegraphics[width=\linewidth]{img/analysis/use_case.pdf}
    \caption{Application's use case diagram}
    \label{fig:use_case}
\end{figure}

As shown on diagram, the application has several use cases

\begin{itemize}
    \item UC1: Displaying nearby cafes as a list or as map view
    \item UC2: From each cafe, it is possible to start navigation
    \item UC3: Each cafe can be marked as a favourite
    \item UC4: Setting filter to filter the cafe results
    \item UC5: User can display favourites cafes
    \item UC6: User can review the cafe's tags
    \item UC7: User can suggest a new tag to the cafe
\end{itemize}

\subsubsection{UC1: Displaying Nearby Cafes}
User can display nearby cafes. The result is filtered by set filter, which can be altered by \textit{C4}

% TODO ---- End Review ---------

\section{Existing alternatives}
The analysis of existing alternatives was made to research already created applications with similar features. Existing applications were searched through Android's official store. Applications with these functionalities were chosen for the review.

\begin{itemize}
    \item Nearby place search
    \item Application's theme should be cafes or similar
\end{itemize}

For comparison five, the most inspiring and distinguish applications were chosen. The following lines briefly describe one of each, their target audience, the advantages and drawbacks. 

\subsection{Gastromapa Lukáše Hejlíka}
\todo{LINK}
Published in the first quarter of 2019 as a new application for exploring restaurants in the Czech Republic.  The application's speciality is that restaurants' reviews are not done by users but by gastronomy specialist \textit{Lukáš~Hejlík}.

As soon as the application launches, it displays nearby restaurants. Each establishment is shown as a card with important information such as an address, distance and type of restaurant. The main card's domain is a large photo which should catch the user's eye to take a look. 

After the card is clicked, the user is presented with the restaurant's detail, where more information such as opening hours, map location and comprehensive review by~\textit{L.~Hejlík} can be found. From this detail navigation to the chosen restaurant can be launched. The target audience is anyone who seeks to visit unknown places and possess the opportunity to taste great food.


\begin{figure}[h]
    \centering
    \includegraphics[width=0.33\linewidth]{img/analysis/gastromapa_hejlik.jpg}
    \caption{Gastromapa \textit{L. Hejlíka} \todo{Include app's images?}}
    \label{fig:gastromapa-hejlik}
\end{figure}

\subsubsection{The advantages}
\begin{itemize}
    \item Design is fresh, clean and users can immediately see relevant content.
    \item Thanks to clean design application is easy to use and understand.
    \item The whole application behaves smoothly without any noticeable freezing.
\end{itemize}

\subsubsection{The drawbacks}
\begin{itemize}
    \item The navigation button has a blackish colour that after scrolling disappears. If the restaurant has a darker photo, the button is hard to notice. 
    \item When coming back to the main screen, loading of the list is started again, and the previous search is lost.
    \item Detail screen on entry is fully covered with restaurant photo. From a design point of view, it is a nice touch, but users must scroll to see any information. 
\end{itemize}

\subsection{Pivní deník}
\todo{LINK} Application \textit{Pivní deník} is used to search nearby pubs in the Czech Republic and their beer offer. The content is created by the community, including served beer and their prices.  Application offers searching nearby restaurants filtered by beer brands. Each user can view a history of places they have visited, furthermore they can mark any pub as their favourite and share their experience.

\begin{figure}[h]
    \centering
    \includegraphics[width=0.33\linewidth]{img/analysis/pivni_denik.png}
    \caption{Pivní deník \todo{Include app's images?}}
    \label{fig:pivni-denik}
\end{figure}

\subsubsection{The advantages}
\begin{itemize}
    \item Pubs are displayed as a list or on the map.
    \item The served brands are displayed directly within the list, so it is not needed to visit details.
    \item The registration is optional for searching. If users want to contribute, they have to have an account.
\end{itemize}

\subsubsection{The drawbacks}
\begin{itemize}
    \item Registration can be done through Facebook or e-mail. With e-mail registration, the user is forced to leave the application and is redirected to the web page where registration is finished.
    \item As was said, content is created by the community. During the research, it was clear that many information is outdated or misleading. 
    \item Overall the application design looks outdated and does not meet current, modern,  trends.
    \item On the primary screen user's stats and at most three nearest pubs are displayed. The drawback is that on the larger screens, there is plenty of unused space. 
    \item Each restaurant displays only one brand of drafted beer. Nowadays, many pubs offer more than one brand. 
    \item Side menu can be opened only with the hamburger icon but not with slide to the right gesture.
\end{itemize}

\subsection{Restu}
\todo{LINK} Restu is another gastronomy guide focused on restaurants in the Czech Republic. Through this application reservations can be made. Application has many unique functionalities. For example ``discover'' section which shows attractive offers or the best cafe in the city. Another functionality is the ``check-in'' button which gives credits to the users if they eat at the given restaurant. Target audience is everyone who searches for new places where to eat and make a reservation.

\begin{figure}[h]
    \centering
    \includegraphics[width=0.33\linewidth]{img/analysis/restu.png}
    \caption{Restu \todo{change image}}
    \label{fig:restu}
\end{figure}

\subsubsection{The advantages}
\begin{itemize}
    \item Clean and well-structured layout.
    \item Opt-in registration.
\end{itemize}

\subsubsection{The drawbacks}
\begin{itemize}
    \item When a restaurant card is selected, window of the restaurant pops up but at the bottom cannot be hidden again.
    \item To review the restaurant, the user has to be signed in and the restaurant must be open. If the restaurant is closed, the review cannot be added.
\end{itemize}

\subsection{Zomato}
\todo{LINK} Zomato is primarily web-based restaurant browser in the world. It has its own database of establishments. Content is edited by users.  
On the primary screen are displayed ``week hits'', top restaurants or ``happy hours''. Restaurants are divided into categories such as ``Nightlife'' or ``Daily menu'' which helps for navigation within the application.
Target audience is anyone who wants to try new restaurants or someone who is looking for action offers. 

\begin{figure}[h]
    \centering
    \includegraphics[width=0.33\linewidth]{img/analysis/zomato.png}
    \caption{Zomato}
    \label{fig:zomato}
\end{figure}

\subsubsection{The advantages}
\begin{itemize}
    \item Well solved filtering. Filter setting is intuitive, displays most used filters. 
    \item Advanced options for filtering with tags such as ``dog friendly'' or ``WiFi free''.
    \item Friendships with other users. If another user added review, notification is received.
\end{itemize}

\subsubsection{The drawbacks}
\begin{itemize}
    \item The primary screen is cluttered with many information at one place.
    \item Nearby restaurants list is hidden below ``favourites restaurants'' and ``month collections''.
    \item Full-text search in some circumstances behaves unexpectedly. For example, to search for restaurants which offer ``asian food'' user has to type exactly ``asian'' but not ``asia''.
    \item Readability of some text is worsened by light background and greyish text colour. In some scenarios, it is hard to read.
\end{itemize}

\subsection{Google maps}
\todo{LINK} Popular worldwide map service by Google. One of the world's biggest database of places, restaurants. 
Google maps for each business, establishment displays additional info such as user reviews, photos, prices. 
Within Android system is already installed. Nearby places can be searched directly from the map. 

\begin{figure}[ht]
    \centering
    \includegraphics[width=0.33\linewidth]{img/analysis/gmaps.png}
    \caption{Google Maps}
    \label{fig:google-maps}
\end{figure}

\subsubsection{The advantages}
\begin{itemize}
    \item Known and tested user interface which is embedded often to another application.
    \item No registration is required.
    \item Place detail includes plenty of useful information.
    \item GPS navigation with one click.
\end{itemize}

\subsubsection{The drawbacks}
\begin{itemize}
    \item Not domain focused, that means it does not offer focused content on particular businesses such as restaurants.
    \item No advanced filtering and result sorting.
\end{itemize}

To the conclusion, five applications were analysed on the Android system. Three apps are focused mainly on gastronomy. Another one specialises on beer. Last one, \textit{Google maps} is one of the most universal and robust. 
Each application has its own unique \gls{ui} \todo{not working gls, why??} and overall user experience differs. In the \cref{table:app-analysis} the targeted audience and user interface usefulness is summarised. \todo{review this}

\begin{table}[htbp]
\centering
\begin{tabular}{|l|l|l|}
\hline
\multicolumn{1}{|c|}{Application} & \multicolumn{1}{c|}{Targeted audience} & \multicolumn{1}{c|}{Overall \gls{ui}} \\ \hline
Gastromapa Lukáše hejlíka         & Everyone                               & Great                           \\ \hline
Pivní deník                       & Beer drinkers                          & Bad.                            \\ \hline
Restu                             & Everyone                               & Bad.                            \\ \hline
Zomato                            & Everyone                               & Good                            \\ \hline
Google                            & Everyone                               & Great but complex              \\ \hline
\end{tabular}
\caption{Analysed applications \gls{ui} summarization}
\label{table:app-analysis}
\end{table}


\section{Application prototype}

% TODO: REVIEW

One crucial step during the creation of software product is prototyping. Prototypes can help introduce different design ideas, can be easily tested, evaluated and changed. Prototyping techniques differ, but the desired output is the same -- provide visually concept of the final product. Prototypes do not help only visually, but they are part of user experience research and can find out which parts of the user interface should be changed before it is implemented.

There is no one correct definition of how prototypes should look and how they should be created. The prototype can be of form as simple sketch on paper to sophisticated pixel-perfect application~\cite{adobe-prototype}. Prototypes can be created multiple times during the whole creation process. 

In the early stages \gls{lofi} prototype is typically created. With \gls{lofi}, the application can be evaluated and user-tested if desired design concept is usable and understandable for users. When \gls{lofi} is finalised, the next prototype~--~\gls{hifi} is created. \gls{hifi} comes out from \gls{lofi} and should behave as fully functional application on the target platform. With \gls{hifi} once again, the application is evaluated with users and tested. 

To be more precise, according to \cite{adobe-prototype} \gls{lofi} prototype is a way to translate high-level concepts into tangible and testable artefacts. The most significant  functionality  of \gls{lofi} prototypes is to check and test the functionality of the product before visual appearance. Advantage of \gls{lofi} is that it is inexpensive, fast way to propose prototype. On the other hand, \gls{lofi} lacks complexity and cannot supply advanced interactivity. \gls{lofi} should be used to quickly create a prototype and get user feedback in the early stages of the creation process. 

After the~\gls{lofi} prototype, the \gls{hifi} is created. This prototype appears and function as similar as possible to the actual built application. \gls{hifi} should be created on the targeted platform and behave as it is the~final product. The goal is to have more complex \gls{ui} interactivity and have better feedback from user testing. Thanks to the fact that prototype looks like a real application user behaves more naturally and can get more precise a meaningful feedback than with \gls{lofi} prototype. 

To conclusion, \gls{lofi} prototypes are tested only internally with a small number of users and can be iterated more often and faster. While \gls{hifi} is more expensive to build and should be created and tested after \gls{lofi} prototype was accepted. On the other hand, \gls{hifi} gives better feedback from user testing, thus provides more valuable information.


\subsection{Coffee Time Prototype}
After the specification was written next step was to create \textit{task list}. Task list is written from the user's perspective that is, each task describe the user's action. It should tackle all important functionalities and even obvious one such as ``add record'' or ``remove record''.

\begin{figure}[htp]
    \centering
    \includegraphics[width=0.9\textwidth]{img/analysis/task-list-graph.png}
    \caption{Coffee Time Task Graph}
    \label{fig:task-graph}
\end{figure}

Because the task list can become very long, it is usually transformed into \textit{task graph}. Task graph does not have any specific definition, but it should contain every task along with each available screen within the application. Coffee Time's task graph is listed in \cref{fig:task-graph}. The blue rectangles are screens and the yellow ones are any task what can users do.

\subsubsection{Low Fidelity Prototype}
\todo{link to used SW?}

\todo{link to clickable pdf?}

\subsubsection{High Fidelity Prototype}
\todo {link to created hifi? GIT-tag}

\subsection{Nielsen heuristic}
\todo{Nielsen heuristic evaluation}

\todo{Use results from MI-NUR. User prototype testing and results}

\todo{domain, architecture, tag state diagram, }

% TODO: \REVIEW ------------------------------------



\section{Technical analysis}

\subsection{Google Places API}
\todo{Google places API and how it is used.}

\section{Serverless approach \todo{Maybe move it to another chapter before analysis?}}

\todo{Firebase. Cloud functions. Firestore - NoSQL document based database. Pricing}
\blind[2]





\chapter{Implementation}
\label{ch:implementation}
\todo{introduction}

\section{High level app overview}
\todo{api, mobile client, express.js, admin SDK, place api}

\section{Coffee Time API}
\todo{express.js, api communication, ...}
\todo{logging,...}

\section{Coffee Time}
\todo{intro}

\todo{the art of clean architecture}

\todo{domain layer}
\todo{equatable}

\todo{data layer}
\todo{json parsing}

\todo{repository layer}
\todo{either functional approach}

\subsection{Representation Layer}
\todo{representation layer}
\todo{Bloc, freezed - union types} 

\todo{screens}
\todo{bloc, states, events,}


\chapter{User testing}
\label{ch:testing}
\todo{Beta user testing. Experiments and results}

\todo{What todo? No real testing done as planned... thank you COVID-19, only a few friends tested app,...}

\todo{Dev process? Github, open issue, Continuous Integration? Open-source?}

\begin{conclusion}


\section{Next Steps}
Coffee Time is for sure not ``feature complete'' and there is plenty of space to improvement. On going future development is planned to obtain more experience with Flutter framework and bring even more better application. 

Next possible features and steps can be:

\begin{itemize}
    \item Add missing feature ``searching cafes in custom location''.
    \item Synchonization of favorited cafes.
    \item Better map view with more information in it.
    \item Optimise application performance, responsivity and adaptability to different form factors, screen sizes and resolutions.
    \item Add full \textit{iOS} support. This includes redesign of the application to more modern cross-platform look. 
    \item As a experiment, build application also for web and desktop platform.
\end{itemize}
	
\end{conclusion}

\bibliographystyle{iso690.bst}
\bibliography{ref.bib}

\appendix

%\chapter{Seznam použitých zkratek}
\printglossaries

\chapter{Content of enclosed CD \todo{USB?}}

Vhodným způsobem vizualizujte obsah přiloženého média. Lze použít balíček \verb|dirtree| a vytvořit např. následující výstup (adresáře src a text s~příslušným obsahem jsou \emph{povinné}):

\begin{figure}
	\dirtree{%
		.1 readme.txt\DTcomment{brief content description}.
		.1 src.
		.2 impl\DTcomment{thesis source code}.
		.2 thesis\DTcomment{thesis text \LaTeX{} source code}.
		.1 text.
		.2 thesis.pdf\DTcomment{Thesis text in PDF form}.
		.1 prototype.
		.2 lofi.pdf\DTcomment{Low Fidelity clickable prototype}.
	}
\end{figure}


\end{document}
