% arara: xelatex
% arara: xelatex
% arara: xelatex

% options:
% thesis=B bachelor's thesis
% thesis=M master's thesis
% czech thesis in Czech language
% english thesis in English language
% hidelinks remove colour boxes around hyperlinks

\documentclass[thesis=M,english]{FITthesis}[2019/12/23]

\usepackage[utf8]{inputenc}

\usepackage{graphicx} %graphics files inclusion
\usepackage{amsmath} %advanced maths
\usepackage{amssymb}
\usepackage{dirtree} %directory tree visualisation
\usepackage{cleveref} % intelligent crossreference
\usepackage{hyperref} % cross-reference
\usepackage{afterpage}
\usepackage{minted} % code highlighting
\usepackage{notoccite} % prevent TOC cite 

\newcommand{\tg}{\mathop{\mathrm{tg}}} %cesky tangens
\newcommand{\cotg}{\mathop{\mathrm{cotg}}} %cesky cotangens

% % list of acronyms
\usepackage[acronym,nonumberlist,toc,numberedsection=autolabel]{glossaries}
\iflanguage{czech}{\renewcommand*{\acronymname}{Seznam pou{\v z}it{\' y}ch zkratek}}{}
\makeglossaries

% following 3 commands (todo, tmp image) Used from http://www.herout.net/blog/2017/03/pomalu-uz-pojdme-psat/
\usepackage{xcolor} 
\newcommand{\todo}[1]{\textcolor{red}{\textbf{[#1]}}}

\usepackage{blindtext}
\newcommand{\blind}[1][1]{\textcolor{gray}{\Blindtext[#1][1]}}

\newcommand{\RNum}[1]{\uppercase\expandafter{\romannumeral #1\relax}}

% % % % % % % % % % % % % % % % % % % % % % % % % % % % % % 

% % % % % % % % % % % % % % % % % % % % % % % % % % % % % % 
% ODTUD DAL VSE ZMENTE
% % % % % % % % % % % % % % % % % % % % % % % % % % % % % % 

\department{Department of software engineering science}
\title{Coffee Time Mobile Application in Flutter}
\authorGN{Petr} %(křestní) jméno (jména) autora
\authorFN{Nymsa} %příjmení autora
\authorWithDegrees{Bc. Petr Nymsa} %jméno autora včetně akademických titulů
\author{Petr Nymsa} %jméno autora bez akademických titulů
\supervisor{Ing. David Šenkýř}
\acknowledgements{Děkuji všem a za vše. Nevíte-li, co sem napsat, příkaz odstraňte.}
\abstractCS{V~několika větách shrňte obsah a přínos této práce v~češtině. Po přečtení abstraktu by měl mít čtenář dost informací pro rozhodnutí, zda chce Vaši práci číst.}
\abstractEN{Sem doplňte ekvivalent abstraktu Vaší práce v~angličtině.}
\placeForDeclarationOfAuthenticity{V~Praze}
\declarationOfAuthenticityOption{4} %volba Prohlášení
\keywordsCS{Závěrečná práce, \LaTeX{}.}
\keywordsEN{Thesis, \LaTeX{}.}
\website{http://site.example/thesis} %volitelná URL práce, objeví se v tiráži


\begin{document}

% Acronyms
\newglossaryentry{formula}
{
        name=formula,
        description={A mathematical expression}
}
\newacronym{ctu}{CTU}{Czech Technical University}

% % % % % % % % % % % % % % % % % % % % % % % % % % 

\begin{introduction}
	Doplňte úvod Vaší práce.  \gls{ctu}
\end{introduction}

\chapter{Nějaká kapitola}

Doplňte kapitoly Vaší práce.

\section{Nějaká sekce}

Doplňte vhodný text.

\chapter{Další kapitola}


\begin{conclusion}
	Doplňte závěr.
	
\end{conclusion}

\bibliographystyle{iso690.bst}
\bibliography{ref.bib}

\appendix

%\chapter{Seznam použitých zkratek}
\printglossaries
% \begin{description}
% 	\item[GUI] Graphical user interface
% 	\item[XML] Extensible markup language
% \end{description}

\chapter{Obsah přiloženého CD}

Vhodným způsobem vizualizujte obsah přiloženého média. Lze použít balíček \verb|dirtree| a vytvořit např. následující výstup (adresáře src a text s~příslušným obsahem jsou \emph{povinné}):

\begin{figure}
	\dirtree{%
		.1 readme.txt\DTcomment{stručný popis obsahu CD}.
		.1 exe\DTcomment{adresář se spustitelnou formou implementace}.
		.1 src.
		.2 impl\DTcomment{zdrojové kódy implementace}.
		.2 thesis\DTcomment{zdrojová forma práce ve formátu \LaTeX{}}.
		.1 text\DTcomment{text práce}.
		.2 thesis.pdf\DTcomment{text práce ve formátu PDF}.
	}
\end{figure}


\end{document}
